%% ======================================================
% TEMPLATE PREAMBLE FOR APG RESEARCH GROUP INTERNAL DOCS ||
% Correspondence author: Gorka Zubia                     ||
% Email: gorka.zubia@ehu.eus			         ||
% First version 29/10/2023			         ||
% =======================================================

\documentclass[10pt, a4paper]{article}


% ------------------------------------------------------------ %
% INCLUDEAK                                             %
% ------------------------------------------------------------ %
% Package with programming tools geared primarily towards LaTeX class and package authors 
\usepackage{etoolbox}       
\usepackage{kvoptions}

% PDFaren testua kopiatu ahal izateko
\usepackage[utf8]{inputenc}

\usepackage{ragged2e}

% Testuaren estuera kontrolatzeko
\usepackage{setspace}

%% Gazteleraz ipintzeko
%\usepackage[spanish,es-tabla]{babel}
%\spanishdecimal{.}
\usepackage[english]{babel}

%% EHUko tipografia ipintzeko
%\usepackage{fontspec}

%% Eranskinak
\usepackage[toc,title,page]{appendix}

% Azpikogidoiak ipintzeko
\usepackage{textcomp}

% Erreferentziak eta estekak ipini ahal izateko
\usepackage[colorlinks,citecolor=blue,linkcolor=blue]{hyperref}

% AKRONIMOAK
\usepackage[nogroupskip,nonumberlist,acronym, nomain,shortcuts]{glossaries-extra}

%% Irudientzako
\usepackage{graphicx}

%% Goiburuentzat
\usepackage{fancyhdr}
\usepackage{xpatch}


%% Marginak eta goiburuen konfiguraziorako
% Allows control of paper geometry
\usepackage{geometry}

% Reiquired for color            
\usepackage[table]{xcolor}

%% Taulentzako
\usepackage{booktabs}

% for setting table width sizes             
\usepackage{tabularx}

% Solution for Latex 2018 error due to chngcntr package update
% https://tex.stackexchange.com/questions/425600/latex-error-command-counterwithout-already-defined?utm_medium=organic&utm_source=google_rich_qa&utm_campaign=google_rich_qa
\let\counterwithout\relax
\let\counterwithin\relax

% to reset counter of figures and tables per section            
\RequirePackage{chngcntr}

%Lerro anitz tauletan
\usepackage{multirow}

% Zutabe anitz tauletan
\usepackage{multicol}

%% Dezimalak alineatzeko tauletan
\usepackage{siunitx}

%% Lengoaia matematikoa erabiltzeko
\usepackage{gensymb}
\usepackage{mathtools}
\usepackage{amssymb}
\usepackage{amsmath}

%% Array ingurunea
\usepackage{array}

% To manage arrays in LaTeX
% IMPORTANT: http://tex.stackexchange.com/questions/65073/latex-doesnt-recognize-endarray
% Errors occure when 'arrayjob' package and 'amsmath' package are used because of \array
% Using 'arrayjobx' resolves the issue by defining \arrayx macro instead of conflicting \array
\usepackage{arrayjobx}

%% Figurak sortzeko
\usepackage{caption}

\usepackage[labelformat=simple]{subcaption}	% Subfigures named like (a), (b), (c)...


%%% To help align tables, figures, pages
\usepackage{float}

% for custom verbatim code definitions - block of code      
\usepackage{fancyvrb}

% for custom verbatim code definitions - inline code            
\usepackage{newverbs}

% Fixes issue with quotes being backwards
\RequirePackage[autostyle, english = american]{csquotes} 

% For enhanced reference capability must be loaded after amsmath package
\usepackage[noabbrev,capitalise]{cleveref}

%% Captions verbatimen
\usepackage{cprotect}

% Used for tracking version history of document
\usepackage[nochapter]{vhistory}  

%% Laukizuzenak sartzeko
\usepackage{framed} %boxed{}
%
%% Testua identatzeko
\usepackage{indentfirst}
%
%% Izenburuentzako
\usepackage{enumitem}

%% Kodea sartzeko
\usepackage{listings}

%% Atalen izenburuentzat
\usepackage{titlesec}

%% Edukinen taularentzako    
\usepackage{titletoc}   

%% Irudi zerrendaren titulurako
\usepackage{tocloft}

% To add appendices to docuemnt
\usepackage[toc]{appendix}

% to list matlab code - formatted
\usepackage[framed,numbered,autolinebreaks]{mcode} 

% For porper text encoding (certain characters dont show up in pdf properly)
\usepackage[T1]{fontenc}

%% Stringen lehenengo karakterea lortzeko
\usepackage{xstring}

% Resolve font encoding errors for missing commands
\usepackage{textcomp}

% For drawing manual pictures - i.e flowcharts, block diagrams, etc...
\usepackage{tikz}

% For creating PDF bookmarks in PDF viewer
\usepackage{bookmark}

% For creating custom line spacing
\usepackage{setspace}

% For adding a large letter to beginning of paragraph 
\usepackage{lettrine}

% To determine last page of each section
\usepackage{lastpage}

\usepackage{refcount}

% To create for loops
\usepackage{multido}

% Additional package that allows for more than 9 symbols for footnotes
\usepackage{alphalph}

% For saythanks command
\usepackage{abstract}

\definecolor{subsectioncolor}{rgb}{0,0.44706,0.40392}


\newcommand{\aref}[1]{the Appendix{\ref{#1}}} % To create the cross-reference of the appendices sections as "Appendix" 
\newcommand{\sref}[1]{\textcolor{black}{Section}~\hyperref[#1]{\ref{#1}}} % To create the cross-reference of the sections as "Section", and make clickable just the number of the section, not "Section" word itself
\newcommand{\ssref}[1]{\textcolor{black}{subsection}~\hyperref[#1]{\ref{#1}}} % To create the cross-reference of the sections as "subsection", and make clickable just the number of the section, not "subsection" word itself
\newcommand{\ssrefs}[1]{\textcolor{black}{subsections}~\hyperref[#1]{\ref{#1}}} % To create the cross-reference of the sections as "subsection", and make clickable just the number of the section, not "subsection" word itself
\newcommand{\sssubref}[1]{\hyperref[#1]{\ref{#1}}} % To create the cross-reference of the sections as "subsection", and make clickable just the number of the section, not "subsection" word itself

\newcommand{\fref}[1]{\textcolor{subsectioncolor}{Fig.}~\hyperref[#1]{\ref{#1}}} 	% To create the cross-reference of the Figures as "Fig.", match its color with the subsection color,and make clickable just the number of the section, not "Fig." word itself
\newcommand{\frefs}[1]{\textcolor{subsectioncolor}{Figs.}~\hyperref[#1]{\ref{#1}}} 	% To create the cross-reference of the Figures as "Figs.", match its color with the subsection color,and make clickable just the number of the section, not "Figs." word itself

\newcommand{\subfref}[1]{\hyperref[#1]{\ref{#1}}} 	% To create the cross-reference of the Figures as "Figs.", match its color with the subsection color,and make clickable just the number of the section, not "Figs." word itself

\newcommand{\tref}[1]{\textcolor{subsectioncolor}{Table}~\hyperref[#1]{\ref{#1}}} 	% To create the cross-reference of the Tables as "Table", match its color with the subsection color,and make clickable just the number of the section, not "Table" word itself
\renewcommand\thesubfigure{(\alph{subfigure})} 										% To appear the subfigure "name" inside parentheses when referenced, for example, Fig. 7(a)

%% Kolorezko kutxak marrazteko
\usepackage{tcolorbox}

%% Koloreak definitu
\definecolor{apgBlueOriginal_face}{RGB}{102,206,245}
\definecolor{apgBlueOriginal_edge}{RGB}{0,  173,239}
\definecolor{mBlue_edge}{RGB}{17,  34,65}
\definecolor{mBlue_face}{RGB}{230,  236,244}

%%%% _____ (1.3.2) Authot ORCIDs ____________________________________________________
\newcommand{\orcidGZ}{\orcidlink{0000-0002-3764-1675}}
\newcommand{\orcidJZ}{\orcidlink{0000-0002-9137-1733}}

%%%%%%%%%%%%%%%%%%%%%%%%%%%%%%%%%%%%%%%%%%%%%%%%%%%%%%%%%%%%%%%%%%%%%%%%%%%%%%%%%%%%%%%%%%%%%%%%%%%
% SECTION: Update tabular rule definitions
%%%%%%%%%%%%%%%%%%%%%%%%%%%%%%%%%%%%%%%%%%%%%%%%%%%%%%%%%%%%%%%%%%%%%%%%%%%%%%%%%%%%%%%%%%%%%%%%%%%

% Command: midrule

\def\midruleNew{\noalign{\ifnum0=`}\fi
  \@aboverulesep=0pt
  \global\@belowrulesep=0pt
  \global\@thisruleclass=\@ne
  \@ifnextchar[{\@BTrule}{\@BTrule[\lightrulewidth]}}

\let\midrule\midruleNew

% Command: toprule

\def\topruleNew{\noalign{\ifnum0=`}\fi
  \@aboverulesep=0pt
  \global\@belowrulesep=0pt
  \global\@thisruleclass=\@ne
  \@ifnextchar[{\@BTrule}{\@BTrule[1.5pt]}}
  
\let\toprule\topruleNew

% Command: bottomrule

\def\bottomruleNew{\noalign{\ifnum0=`}\fi
  \@aboverulesep=0pt
  \global\@belowrulesep=0pt
  \global\@thisruleclass=\@ne
  \@ifnextchar[{\@BTrule}{\@BTrule[1.5pt]}}
  
\let\bottomrule\bottomruleNew
%%%%%%%%%%%%%%%%%%%%%%%%%%%%%%%%%%%%%%%%%%%%%%%%%%%%%%%%%%%%%%%%%%%%%%%%%%%%%%%%%%%%%%%%%%%%%%%%%%%
% SECTION: Common Definitions
%%%%%%%%%%%%%%%%%%%%%%%%%%%%%%%%%%%%%%%%%%%%%%%%%%%%%%%%%%%%%%%%%%%%%%%%%%%%%%%%%%%%%%%%%%%%%%%%%%%

 % Define common colors
\definecolor{input}{RGB}{61, 98, 233}

% % Formatter for default values fo type input
\newcommand{\varInput}[2]{{\normalfont\footnotesize\textbf{\color{input}\textbackslash{#1}}[{\normalfont#2}]}}
\newcommand{\authorInput}[2]{\normalfont\footnotesize\textbf{\color{input}\textbackslash{#1}}=\{{\normalfont#2}\}}
%%%%%%%%%%%%%%%%%%%%%%%%%%%%%%%%%%%%%%%%%%%%%%%%%%%%%%%%%%%%%%%%%%%%%%%%%%%%%%%%%%%%%%%%%%%%%%%%%%%
% SECTION: Document Author Parameters
%%%%%%%%%%%%%%%%%%%%%%%%%%%%%%%%%%%%%%%%%%%%%%%%%%%%%%%%%%%%%%%%%%%%%%%%%%%%%%%%%%%%%%%%%%%%%%%%%%%

% Define document authors array and default values
\newarray\docAuthorName
\readarray{docAuthorName}{\authorInput{docAuthorName(1)}{name}&\authorInput{docAuthorName(2)}{name}&&&&&&&&}

% Define document author positions array and default values
\newarray\docAuthorPosition
\readarray{docAuthorPosition}{\authorInput{docAuthorPosition(1)}{input}&\authorInput{docAuthorPosition(2)}{input}&&&&&&&&}

% Define document author organizations array and default values
\newarray\docAuthorOrganization
\readarray{docAuthorOrganization}{\authorInput{docAuthorOrganization(1)}{input}&\authorInput{docAuthorOrganization(2)}{input}&&&&&&&&}

% Define document author locations array and default values
\newarray\docAuthorLocation
\readarray{docAuthorLocation}{\authorInput{docAuthorLocation(1)}{input}&\authorInput{docAuthorLocation(2)}{input}&&&&&&&&}

\newcommand{\docAuthor}[5][99]{
   \docAuthorName(#1)={#2}
   \docAuthorPosition(#1)={#3}
   \docAuthorOrganization(#1)={#4}
   \docAuthorLocation(#1)={#5}
   \def\@docAuthor{}
}\docAuthor{}{}


%%%%%%%%%%%%%%%%%%%%%%%%%%%%%%%%%%%%%%%%%%%%%%%%%%%%%%%%%%%%%%%%%%%%%%%%%%%%%%%%%%%%%%%%%%%%%%%%%%%
% SECTION: Document Style Parameters
%%%%%%%%%%%%%%%%%%%%%%%%%%%%%%%%%%%%%%%%%%%%%%%%%%%%%%%%%%%%%%%%%%%%%%%%%%%%%%%%%%%%%%%%%%%%%%%%%%%

\newcommand{\docRefRangeConjunction}[1][--]{\def\@docRefRangeConjunction{#1}}\docRefRangeConjunction{}
% Defines how fussy LaTeX is about underfull or overfull boxes (horizontal and vertical )
\hbadness=2000
\vbadness=2000

%%%%%%%%%%%%%%%%%%%%%%%%%%%%%%%%%%%%%%%%%%%%%%%%%%%%%%%%%%%%%%%%%%%%%%%%%%%%%%%%%%%%%%%%%%%%%%%%%%%
% SECTION: General formatting definitions
%%%%%%%%%%%%%%%%%%%%%%%%%%%%%%%%%%%%%%%%%%%%%%%%%%%%%%%%%%%%%%%%%%%%%%%%%%%%%%%%%%%%%%%%%%%%%%%%%%%

% Define base geometry
\geometry{margin=1in,includefoot,marginparwidth=0in} 

% Define graphics inputs
\DeclareGraphicsExtensions{.pdf,.png,.jpg,.ps,.gif,.eps}

% Remove white space caused by using the various package environments
\AtBeginEnvironment{center}{\vspace{-\baselineskip}}
\AtEndEnvironment{center}{\vspace{-\baselineskip}}

% Command to fix issue with outter quotation marks being the wrong direction
\MakeOuterQuote{"}

% Custom title formatting - to force new line for paragraph and subparagraph
\titleformat{\paragraph}[hang]{\normalfont\normalsize}{\theparagraph}{1em}{}
\titleformat{\subparagraph}[hang]{\normalfont\normalsize\itshape}{\thesubparagraph}{1em}{}

% Make figure, table, and equation captions bold
\captionsetup[table]{font={bf,normalsize},position=above,skip=0.5\baselineskip}
\captionsetup[figure]{font={bf,normalsize},position=below}
\captionsetup[equation]{font={bf,normalsize},position=auto}

% Add new table columns 
\newcolumntype{L}[1]{>{\raggedright\let\newline\\\arraybackslash\hspace{0pt}}m{#1}}
\newcolumntype{C}[1]{>{\centering\let\newline\\\arraybackslash\hspace{0pt}}m{#1}}
\newcolumntype{R}[1]{>{\raggedleft\let\newline\\\arraybackslash\hspace{0pt}}m{#1}}

% Define the caption skip parameter
\setlength{\belowcaptionskip}{-0.25\baselineskip}

% Change list structure spacing
\setlist[itemize]{itemsep = 0pt}
\setlist[enumerate]{itemsep = 0pt}
\setlist[description]{itemsep = 0pt}

% Set up hyperlink options
\hypersetup{hidelinks}
\hypersetup{urlcolor=cyan}

% Define the maximum section depth
\setcounter{secnumdepth}{6}

%% ------------------------------------------------------------ %
%%                         AURKIBIDEA                           %
%% ------------------------------------------------------------ %

\newcommand{\printTOC}{
      % Create the TOC
      \newpage
      \setlength{\parskip}{0.4\baselineskip}
      % Add a PDF bookmark in PDF viewer 
      \phantomsection
      \renewcommand\contentsname{Table of Contents}
      \pdfbookmark[2]{Table of Contents}{ToC}
      %Create the ToC
      \begingroup
      \hypersetup{linkcolor=black}
      \tableofcontents
      \endgroup
      \setlength{\parskip}{\baselineskip}
   }
\newcommand{\printLOT}{
      % Create a Table of Contents Entry for LOT
      \newpage
      \setlength{\parskip}{0.4\baselineskip}
      \phantomsection
      \bookmarksetupnext{level=subsection}
      \addcontentsline{toc}{section}{List of Tables}
      % Create LOT
      \listoftables
      \setlength{\parskip}{\baselineskip}
}

\newcommand{\printLOF}{
      % Create a Table of Contents Entry for LOF
      \newpage
      \setlength{\parskip}{0.4\baselineskip}
      \phantomsection
      \bookmarksetupnext{level=subsection}
      \addcontentsline{toc}{section}{List of Figures}
      % Create LOF
      \listoffigures
      \setlength{\parskip}{\baselineskip}
}

\newcommand{\printLOA}{
      % Change the glossary title to nothing, we will create our own
      \renewcommand{\glossarysection}[2][]{}
      \newpage
      \setlength{\parskip}{0.25\baselineskip}
      \phantomsection
      \bookmarksetupnext{level=subsection}
      \addcontentsline{toc}{section}{Acronyms}
      \section*{Acronyms}
      \printglossary[style=alttree,type=\acronymtype]
      \setlength{\parskip}{\baselineskip} 
}


%%% ------------------------------------------------------------ %
%%            TXANTILIOIAREN ITURRIAK              %
%% ------------------------------------------------------------ %
%%\setmainfont[
%% Path		= 2_Fonts/,
%% Ligatures	= TeX,
%% UprightFont	= EHUSans-Light.otf,
%% BoldItalicFont	= EHUSans-BoldItalic.otf,
%% BoldFont      	= EHUSans-Bold.otf,
%% ItalicFont    	= EHUSans-LightItalic.otf
%% ]{EHUSans}

%___________Dokumentuaren inizialak lortu______
\def\firstletters{\bgroup \catcode`-=10 \catcode`(=10 \filA}
\def\filA#1{\filB#1 {\end} }
\def\filB#1#2 {\ifx\end#1\egroup \else#1\expandafter\filB\fi} 

\edef\InizialakDocTitle{%
	\noexpand\firstletters{\docTitle}
}

\edef\InizialakDocTypeTitle{%
	\noexpand\firstletters{\docTypeTitle}
}

% %%%%%%%%%%%%%%%%%%%%%%%%%%%%%%%%%%%%%%%%%%%%%%%%%%%%%%%%%%%%%%%%%%%%%%%%%%%%%%%%%%%%%%%%%%%%%%%%%%%%
%% TEMPLATE: Define document header and footer settings
%%%%%%%%%%%%%%%%%%%%%%%%%%%%%%%%%%%%%%%%%%%%%%%%%%%%%%%%%%%%%%%%%%%%%%%%%%%%%%%%%%%%%%%%%%%%%%%%%%%%
% Set the geometry for each page
\geometry{head=0.3in,foot=0.25in,top=0.2in,left=1in,right=1in,bottom=0.95in,footskip=0.35in,includeheadfoot}
%\geometry{head=0.5in,foot=0.25in,top=0.65in,left=1in,right=1in,bottom=0.95in,footskip=0.35in,includeheadfoot}
% Set new paragraph indentation to zero
\setlength{\parindent}{0in}

% Define paragraph skip sizes - space between paragraphs
\setlength{\parskip}{12pt}

% Create a section break after each "section"
\let\oldsection\section
\renewcommand\section{\clearpage\oldsection}

% Define that figures and tables are numbered based on section e.g. "Table X.X"
\counterwithin{figure}{section}
\counterwithin{table}{section}
\counterwithin{equation}{section}

% Define spacing between Section titles and text
\titlespacing{\section}{0pt}{0.25\baselineskip}{0\baselineskip}
\titlespacing{\subsection}{0pt}{0.25\baselineskip}{0\baselineskip}
\titlespacing{\subsubsection}{0pt}{0.25\baselineskip}{0\baselineskip}
\titlespacing{\paragraph}{0pt}{0.25\baselineskip}{0\baselineskip}
\titlespacing{\subparagraph}{0pt}{0.25\baselineskip}{0\baselineskip}

% Note: counterwithin for lstlisting needs to happen after \begin{document}
%\AtBeginDocument{\counterwithin{lstlisting}{section}}

% Define spacing between Section titles and text
\titlespacing{\section}{0pt}{0.25\baselineskip}{0\baselineskip}
\titlespacing{\subsection}{0pt}{0.25\baselineskip}{0\baselineskip}
\titlespacing{\subsubsection}{0pt}{0.25\baselineskip}{0\baselineskip}
\titlespacing{\paragraph}{0pt}{0.25\baselineskip}{0\baselineskip}
\titlespacing{\subparagraph}{0pt}{0.25\baselineskip}{0\baselineskip}

 % FRONT & MAINMATTER

\makeatletter
\newif\if@mainmatter
\newcommand{\frontmatter}{%
  \clearpage
  \pagenumbering{roman}
  \edef\computelastpage{%
    \romannumeral\numexpr\getpagerefnumber{LastFrontPage}-1\relax}}
\newcommand{\mainmatter}{%
  \clearpage
  \immediate\write\@auxout{\noexpand\newlabel{LastFrontPage}{{}{\arabic{page}}}}%
  \@mainmattertrue
  \pagenumbering{arabic}
  \def\computelastpage{\pageref{LastPage}}}
  \makeatother

% Remove space after figure and table environments
\AfterEndEnvironment{table}{\vspace{-1\baselineskip}}
\AfterEndEnvironment{figure}{\vspace{-1\baselineskip}}
\AtBeginEnvironment{quote}{\vspace{-\baselineskip}}
\AtEndEnvironment{quote}{\vspace{-\baselineskip}}
\AtBeginEnvironment{center}{\vspace{-0.5\baselineskip}}
\AtEndEnvironment{center}{\vspace{-0.5\baselineskip}}
\AtBeginEnvironment{itemize}{\vspace{-0.5\baselineskip}}
\AtEndEnvironment{itemize}{\vspace{-0.5\baselineskip}}
\AtBeginEnvironment{enumerate}{\vspace{-0.5\baselineskip}}
\AtEndEnvironment{enumerate}{\vspace{-0.5\baselineskip}}

% Define the headheight - set to zero, no header in this format
\setlength{\headheight}{60pt}

% % Modify the header and footer lines - set to zero to remove
\renewcommand{\headrulewidth}{0pt}
\renewcommand{\footrulewidth}{1pt}

\newcommand{\TitlePageGeometry}{\newgeometry{top=0.75in,left=1in,right=1in,bottom=0.75in}}

%%%%%%%%%%%%%%%%%%%%%%%%%%%%%%%%%%%%%%%%%%%%%%%%%%%%%%%%%%%%%%%%%%%%%%%%%%%%%%%%%%%%%%%%%%%%%%%%%%%
% TEMPLATE: Define document header and footer settings
%%%%%%%%%%%%%%%%%%%%%%%%%%%%%%%%%%%%%%%%%%%%%%%%%%%%%%%%%%%%%%%%%%%%%%%%%%%%%%%%%%%%%%%%%%%%%%%%%%%

% Define size of the header/footer text
\newcommand{\hfsize}{\footnotesize}

% Create the Footer - Clear the current header/footer style
\fancyhf{} 

% Set the current header/footer style
\pagestyle{fancy}

% Define the centered header
\chead{
\begin{tabular}{|L{.58\textwidth}|p{.365\textwidth}|}
   \hline
   \multicolumn{2}{|c|}{
      
         %\hfsize\bfseries\docTypeTitle~-~\docTitle
      \hfsize\bfseries \InizialakDocTypeTitle - \InizialakDocTitle
   } \\
   \hline
   \multirow{2}{*}{\parbox{.58\textwidth}{\hfsize\hangindent=0.36in \raggedright Title: \docTitle}}
      & \hfsize  Document No: \docNumber \\
      & \hfsize  Effective Date: \docDate \\
   \hline
\end{tabular}
\vspace{-22pt}
}%
%% Left footer
\lfoot{\footnotesize\textbf{\docFooter}}
%%
%%% Right Footer
\rfoot{\hfsize Page: \thepage \hspace{2pt} of \hspace{2pt} \begingroup\hypersetup{linkcolor=black}\computelastpage\endgroup}
%___________Irudien erroa_________________
\graphicspath{{2_figs/}}

%%%%%%%%%%%%%%%%%%%%%%%%%%%%%%%%%%%%%%%%%%%%%%%%%%%%%%%%%%%%%%%%%%%%%%%%%%%%%%%%%%%%%%%%%%%%%%%%%%%
% TEMPLATE: Settings for TOC, LOF, LOT, Acronyms
%%%%%%%%%%%%%%%%%%%%%%%%%%%%%%%%%%%%%%%%%%%%%%%%%%%%%%%%%%%%%%%%%%%%%%%%%%%%%%%%%%%%%%%%%%%%%%%%%%%

% This command adds dots in TOC for all entries (sections didnt have them)
\renewcommand{\cftsecleader}{\cftdotfill{\cftdotsep}}
%\renewcommand{\cftdot}{} % Kendu puntutxuak aurkibidean
\renewcommand{\cftsecleader}{\cftdotfill{\cftdotsep}} % Gehitu puntutxuak TOC sarrera guztientzako (sections ez zeukaten)
\renewcommand\cftdotsep{1.5} 			      % Puntutxuen arteko espazioa
% Identazioa
% Set the table of contents level depth
\setcounter{tocdepth}{4}

% Set the indent spacing
\setlength{\cftsecindent}{0cm}
\setlength{\cftsubsecindent}{0.5cm}
\setlength{\cftsubsubsecindent}{1cm}
\setlength{\cftparaindent}{1.5cm}
\setlength{\cftsubparaindent}{2cm}

% Create Table of Contents, ToC
% Change font sizes in ToC
\renewcommand\cftsecfont         {\fontsize{11pt}{6pt}\selectfont\bfseries}
\renewcommand\cftsubsecfont      {\fontsize{10pt}{6pt}\selectfont}
\renewcommand\cftsubsubsecfont   {\fontsize{10pt}{6pt}\selectfont}
\renewcommand\cftparafont        {\fontsize{10pt}{6pt}\selectfont}
\renewcommand\cftsubparafont     {\fontsize{10pt}{6pt}\selectfont}

% Add some spacing so the section number and title dont overlap in TOC
\setlength\cftsecnumwidth{1\baselineskip}
\setlength\cftsubsecnumwidth{1.5\baselineskip}
\setlength\cftsubsubsecnumwidth{2\baselineskip}
\setlength\cftparanumwidth{2.5\baselineskip}
\setlength\cftsubparanumwidth{3\baselineskip}
\setlength{\cftbeforetoctitleskip}{1in}

% Change font sizes in LoF and LoT
\renewcommand\cftfigfont         {\fontsize{10pt}{6pt}\selectfont}
\renewcommand\cfttabfont         {\fontsize{10pt}{6pt}\selectfont}

% Add some spacing so the figure or table number and title dont overlap in LOF ot LOT
\setlength\cftfignumwidth{35pt}
\setlength\cfttabnumwidth{35pt}
\setlength\cftfigindent{0pt}
\setlength\cfttabindent{0pt}
\setlength\cftbeforefigskip{0pt}
\setlength\cftbeforetabskip{0pt}

%%%% Eranskinen erreferentzia
%%\newcommand{\aref}[1]{\hyperref[#1]{Anexo~\ref{#1}}}
%
%%% verb egiteko listingen captionetan
%\protected\def\psverb#1{\def\innerpsverb##1#1{\texttt{##1}}\innerpsverb}

%%%%%%%%%%%%%%%%%%%%%%%%%%%%%%%%%%%%%%%%%%%%%%%%%%%%%%%%%%%%%%%%%%%%%%%%%%%%%%%%%%%%%%%%%%%%%%%%%%%
% SECTION: Main Report Section
%%%%%%%%%%%%%%%%%%%%%%%%%%%%%%%%%%%%%%%%%%%%%%%%%%%%%%%%%%%%%%%%%%%%%%%%%%%%%%%%%%%%%%%%%%%%%%%%%%%

%\AtBeginDocument{
%   
%   % Create the glossaries for citations and acronyms
%   \makeglossaries 
%
%   % Define glossary style
%   \renewcommand{\glsnamefont}[1]{\hspace{-0.5\parindent}\textbf{#1}} % Set aronyms back to bold
%   \renewcommand{\glossarypreamble}{\normalsize} % Change fontsize
%   \renewcommand{\arraystretch}{1.2}
%   \setlength{\glspagelistwidth}{1.2\textwidth}
%   \setlength{\glsdescwidth}{\textwidth}
%   \glssetwidest[0]{ALongPlaceHolder}
%
%   % Reset the glossary entries (acronyms) so that they are re-defined after the executive summary or synopsis
%   \glsresetall
%}

% Make the list of acronyms
\renewcommand{\glsnamefont}[1]{\textbf{#1}} % Set aronyms back to bold
\renewcommand{\glossarypreamble}{\small} % Change fontsize
\renewcommand{\arraystretch}{1.2}
\setlength{\glspagelistwidth}{0.8\textwidth}
\setlength{\glsdescwidth}{0.8\textwidth}
\glssetwidest[0]{ALongPlaceHolder}

%%%%%%%%%%%%%%%%%%%%%%%%%%%%%%%%%%%%%%%%%%%%%%%%%%%%%%%%%%%%%%%%%%%%%%%%%%%%%%%%%%%%%%%%%%%%%%%%%%%
% TEMPLATE: Update the settings for the appendix
%%%%%%%%%%%%%%%%%%%%%%%%%%%%%%%%%%%%%%%%%%%%%%%%%%%%%%%%%%%%%%%%%%%%%%%%%%%%%%%%%%%%%%%%%%%%%%%%%%%

% This adds the \backmatter to the begining of appendices to change numbering
\AtBeginEnvironment{appendices}{
   % For page numbering - back to roman
   % \backmatter

   % For subsection numbering - use alphanumeric combination
   \renewcommand{\thesubsection}{\thesection.\arabic{subsection}}

   % This helps to setup pdf bookmarks for appendices
   \titleformat{\section}{\normalfont\Large\bfseries\bookmarksetupnext{level=subsection}}{\thesection}{1em}{}
   \titleformat{\subsection}{\normalfont\large\bfseries\bookmarksetupnext{level=subsubsection}}{\thesubsection}{1em}{}
   \titleformat{\subsubsection}{\normalfont\normalsize\bfseries\bookmarksetupnext{level=paragraph}}{\thesubsubsection}{1em}{}
   }

%%%%%%%%%%%%%%%%%%%%%%%%%%%%%%%%%%%%%%%%%%%%%%%%%%%%%%%%%%%%%%%%%%%%%%%%%%%%%%%%%%%%%%%%%%%%%%%%%%%
% TEMPLATE: Define custom content commands to be used in template file(s)
%%%%%%%%%%%%%%%%%%%%%%%%%%%%%%%%%%%%%%%%%%%%%%%%%%%%%%%%%%%%%%%%%%%%%%%%%%%%%%%%%%%%%%%%%%%%%%%%%%%

\newcommand{\printRevisionLog}{
      \newpage \null
      \phantomsection
      \pdfbookmark[2]{Revision Log}{revision}

      \begin{center}
         \textbf{REVISION LOG}
      \end{center}

      \vspace{0.25in}

              	\docRevisionLog
    
       }

\newcommand{\printAuthorPage}{
	\checkdocAuthorName(1)
	\ifemptydata
  	% If not input for docContributorName(1), skip entire section
	\else
   	\newpage \null
  	 \phantomsection
 	  \pdfbookmark[2]{Author Page}{author}

  	 \begin{center}
  	  	\textbf{AUTHOR PAGE}
  	 \end{center}

  	 \textbf{Prepared by:} \vspace{6pt}

  	 \multido{\ii=1+2}{10}{%
   	 \checkdocAuthorName(\ii)
   	 \ifemptydata
      	 % \docAuthor not defined for this index, skip
    	\else
       	\begin{minipage}{.48\textwidth}%
	%       \ifdefstring{\@docSigned}{true}{
        	  \textit{/s/ \docAuthorName(\ii)} \hfill \docSignDate 
	%       }{
	%          \textit{\small To Sign:} 
	%          {\small\textnormal{\color{input}\textbackslash duocSignuued[{\color{black} true}]}}
	%          {\small\textnormal{\color{input}\textbackslash docSignDauuute[{\color{black} date}]}}  
	%       }  
       		\vspace{6pt} \hrule \vspace{5pt}
       		\small\textbf{\docAuthorName(\ii)} \hfill \textbf{Date} \break
	%       \small\textbf{\docAuthorName(\ii)} \hfill \break
       		\docAuthorPosition(\ii) \hfill \break
       		\docAuthorOrganization(\ii) \hfill \break
       		\docAuthorLocation(\ii) \hfill 
       	\end{minipage}%
       \begin{minipage}{.01\textwidth}%
       		\hfill
       \end{minipage}%
       \edef\iinext{\number\numexpr\ii+1\relax}
       \checkdocAuthorName(\iinext)
       \ifemptydata
          \vspace{12pt}
          \break
       \else
          \begin{minipage}{.48\textwidth}%
%          \ifdefstring{\@docSigned}{true}{
             \textit{/s/ \docAuthorName(\iinext)} \hfill \docSignDate 
  %        }{
  %          \textit{\small To Sign:} 
  %           {\small\textnormal{\color{input}\textbackslash douuucSigned[{\color{black} true}]}}
  %          {\small\textnormal{\color{input}\textbackslash docSignDate[{\color{black} date}]}} 
  %        }  
            \vspace{6pt} \hrule \vspace{5pt}
            \small\textbf{\docAuthorName(\iinext)} \hfill \textbf{Date} \break
            \docAuthorPosition(\iinext) \hfill \break
            \docAuthorOrganization(\iinext) \hfill \break
            \docAuthorLocation(\iinext) \hfill 
            \end{minipage}%
            \hfill
            \vspace{12pt}
       \fi    
    \fi   
   }
\fi     
}

% Header and footer in TOC/LOF/LOT
\fancypagestyle{plain}{}

%%%%%%%%%%%%%%%%%%%%%%%%%%%%%%%%%%%%%%%%%%%%%%%%%%%%%%%%%%%%%%%%%%%%%%%%%%%%%%%%%%%%%%%%%%%%%%%%%%%
% SECTION: Create all the front matter prior to user content
%%%%%%%%%%%%%%%%%%%%%%%%%%%%%%%%%%%%%%%%%%%%%%%%%%%%%%%%%%%%%%%%%%%%%%%%%%%%%%%%%%%%%%%%%%%%%%%%%%%

\AtBeginDocument{

   % Add a PDF bookmark in PDF viewer for all of front matter
   \pdfbookmark[1]{Front Matter}{front}

   \frontmatter

   % Add cover sheet only if defined
   \newpage \clearpage \thispagestyle{empty}
   \addtocounter{page}{-1}
   
   \phantomsection
   \pdfbookmark[2]{Title Page}{titlepage}
   \TitlePageGeometry

   % Create graphics at the top of the page
   \begin{minipage}{0.25\textwidth}
      \begin{flushleft}
        \includegraphics[width=\textwidth]{\docLogo}
      \end{flushleft}
   \end{minipage}%
   \begin{minipage}{0.7\textwidth}
      \RaggedLeft\textbf{\large \docTypeTitle \\ \vspace{10pt} Technical Report}
   \end{minipage}%

   \vspace{\baselineskip}
   {\color{black}\hrule height 2pt}

   \vspace{0.5in}

   \raggedright
   
   \begin{flushleft}
      \textbf{\Large \docTitle} 
      \setstretch{1.8}
   \end{flushleft}

   \vspace{24pt}

   \textit{\large Document Number: \docNumber} 

   \vspace{1in}

   \textbf{\large Version: \docVersion}   % The current version numbe, the version number of the last entry—can be read with the command,\vhCurrentVersion
   
   \vspace{12pt}  
   
   {\large \docDate}

    % fill in empty space
   \vfill

   \includegraphics[height=1in]{\docOrgLogo}

   \vspace{-6pt}

   \begin{flushleft}

      \setlength{\parskip}{0pt}

      \docOrganization

   \end{flushleft}

   \vspace{0.1in} 

   \begin{center}
      \footnotesize\textbf{\docFooter}
   \end{center}

   \vspace{0.15in} 

   % Change the geometry back
   \restoregeometry

   % Print the revision log page
   \printRevisionLog

   % Print executive summary
   %\printExecutiveSummary

   % Print author page
   \printAuthorPage

   % Print contributor page
   %\printContributorPage

   % Print the signature page
   %\printApproverPage
   
   % Create the Table of Contents
   \printTOC

   % Create the List of Tables
   %\printLOT

   % Create the List of Figures
   %\printLOF

   % Create the List of Listings
   %\printLOL

   % Create the List of Acronyms
   %\printLOA

   % Create the Nomenclature definition section
   %\printNomenclature

   % remove padding to row height as to no affect other tables
   \setlength\extrarowheight{0pt} 

   \mainmatter

   % Make all text justified and normalsize
   \justifying\normalsize

   % Set paragraph indent to zero
   \setlength{\parindent}{0pt}

}
